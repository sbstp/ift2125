\documentclass[10pt,letterpaper]{article}
\usepackage[utf8]{inputenc}
\usepackage{setspace}
\usepackage[parfill]{parskip}
\usepackage{fullpage}
\usepackage{dirtytalk}
\usepackage{amsfonts}
\usepackage{mathtools}
\usepackage{listings}

\pagenumbering{gobble}

\newenvironment{mymatrix}
{\left[ \begin{array}{ccc|c}}
{\end{array} \right]}

\begin{document}
	Gabriel Lemyre \& Simon Bernier St-Pierre \hfill Devoir 2\\
	\section*{Question 1}
	\begin{enumerate}
		\item[a)]
		$$
		T(n) =
		\begin{cases}
		2, & 0 \le n \le 1\\
		3T(\frac{n}{3}) + 2T(\frac{n}{3}) + 16 \log_3 n & n > 1 \\
		\end{cases}
		$$
		\begin{enumerate}
			\item[(1)] $$\text{On replace } n \text{ par } 3^k$$
			\item[(2)] $$t_k = T(3^k)$$
			\item[(3)]
			$$t_k = 3T(3^{k-1}) + 2T(3^{k-1}) + 16 \log_3 3^k$$
			$$t_k = 5T(3^{k-1}) + 16k$$
			$$t_k = 5t_{k-1} + 16k$$
			\item[(4)]
			$$t_k - 5t_{k-1} = 16k$$
			$$(x-5)(x-1)^2$$
			$$r_1 = 5, r_2 = 1$$
			$$t_k = c_15^k + c_21^k + c_3k1^k$$
			\item[(5)]
			$$T(3^k) = tk \text{ donc, } T(n) = t \log_3 n$$
			$$T(n) = c_15^{log_3 n} + c_21^{log_3 n} + c_3 \log_3 n1^{log_3 n}$$
			\item[(6)]
			$$T(n) \in O(5^{log_3 n})$$
			\item[(7)]
			$$T(1) = 2$$
			$$T(3) = 3T(3) + 16 \log_3 3 = 10 + 16 = 26$$
			$$T(4) = 5T(3) + 16 \log_3 9 = 130 + 32 = 162$$
			
			\[
			\begin{mymatrix}
			1 & 1 & 0 & 2\\
			5 & 1 & 1 & 26\\
			25 & 1 & 2 & 162\\
			\end{mymatrix}
			\rightarrow
			\begin{mymatrix}
			1 & 1 & 0 & 2\\
			5 & 1 & 1 & 26\\
			0 & -4 & -3 & 32\\
			\end{mymatrix}
			\rightarrow
			\begin{mymatrix}
			1 & 1 & 0 & 2\\
			5 & 1 & 1 & 26\\
			0 & 4 & 3 & -32\\
			\end{mymatrix}
			\rightarrow
			\]
			
			\[
			\begin{mymatrix}
			1 & 1 & 0 & 2\\
			4 & 0 & 1 & 24\\
			0 & 4 & 3 & -32\\
			\end{mymatrix}
			\rightarrow
			\begin{mymatrix}
			1 & 1 & 0 & 2\\
			4 & 0 & 1 & 24\\
			4 & 4 & 4 & -8\\
			\end{mymatrix}
			\rightarrow
			\begin{mymatrix}
			1 & 1 & 0 & 2\\
			4 & 0 & 1 & 24\\
			0 & 0 & 4 & -16\\
			\end{mymatrix}
			\rightarrow
			\]
			
			\[
			\begin{mymatrix}
			1 & 1 & 0 & 2\\
			4 & 0 & 1 & 24\\
			0 & 0 & 1 & -4\\
			\end{mymatrix}
			\rightarrow
			\begin{mymatrix}
			1 & 1 & 0 & 2\\
			4 & 0 & 0 & 28\\
			0 & 0 & 1 & -4\\
			\end{mymatrix}
			\rightarrow
			\begin{mymatrix}
			1 & 1 & 0 & 2\\
			1 & 0 & 0 & 7\\
			0 & 0 & 1 & -4\\
			\end{mymatrix}
			\rightarrow
			\]
			
			\[
			\begin{mymatrix}
			0 & 1 & 0 & -5\\
			1 & 0 & 0 & 7\\
			0 & 0 & 1 & -4\\
			\end{mymatrix}
			\rightarrow
			\begin{mymatrix}
			1 & 0 & 0 & 7\\
			0 & 1 & 0 & -5\\
			0 & 0 & 0 & -4\\
			\end{mymatrix}
			\]
			\item[(8)]
			$$T(n) = 7 \times 5^{\log_3 n} -5 \times 1^{\log_3 n} -4 \log_3 n \times 1^{\log_3 n}$$
			$$= 7 \times 5^{\log_3 n} - 4{\log_3 n} - 5$$
		\end{enumerate}
		
		\item[b)]
		
		\begin{enumerate}
			\item[(i)] 
			$c_1$ est positif, on a donc l'ordre exact. $c_1$ était la constante la plus significative.
			$$T(n) \in \Theta(5^{log_3 n})$$
			\item[(ii)] En utilisant du ``Intelligent guesswork'' et la ``smoothness rule'', on peut arriver à la même réponse.
		\end{enumerate}
	\end{enumerate}

	\section*{Question 2}

	\section*{Question 3}
	On trie les chefs par temps utilisé dans la salle B en ordre décroissant. Ainsi, les chefs qui prennent
	beaucoup de temps dans la salle B passeront en premier, alors que les chefs qui passent peu de temps dans la salle B attenderons après les autres. On peut se permettre ce fonctionnement parce que la salle B ne limite pas le nombre de chefs, alors que la salle A demande l'exclusivité. Un algo simple en Python qui produit l'horaire des chefs par numéro est disponible ci dessous.
	\begin{lstlisting}[language=Python]
def horaire(n: int) -> List[int]:
	h = list(range(n))
	h.sort(key=lambda i: c[i] + d[i], reverse=True)
	return h
	\end{lstlisting}	
	\section*{Question 4}
\end{document}