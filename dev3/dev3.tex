\documentclass[10pt,letterpaper]{article}
\usepackage[utf8]{inputenc}
\usepackage{setspace}
\usepackage[parfill]{parskip}
\usepackage{fullpage}
\usepackage{dirtytalk}
\usepackage{amsfonts}
\usepackage{mathtools}
\usepackage{listings}
\usepackage{courier}

\lstset{basicstyle=\ttfamily}

\newenvironment{mymatrix}
{\left[ \begin{array}{cc|c}}
	{\end{array} \right]}


\begin{document}
	Gabriel Lemyre \& Simon Bernier St-Pierre \hfill Devoir 2\\
	\section*{Question 1}
	La récurrence de \lstinline|mergesort4| est
	$$T(n) = 4T(\frac{n}{4}) + \Theta(n)$$
	car à chaque étape on divise la liste en 4, et pour chaque sous liste on appèle à nouveau \lstinline|mergesort4|. À chaque étape, on doit assembler les sous-listes en une seule liste, chose qui se fait linéairement, d'où $\Theta(n)$.
	
	On peut résoudre cette récurrence facilement avec le \textit{master theorem},
	$$a = 4, b = 4, f(n) = n, c = 1$$
	$$\log_4 4 = c$$
	En utilisant le cas 2 du théorème,
	$$t(n) = O(n\log n)$$
	
	Pour $T(n)$, on fait une substitution $T(n) \le cn\log n$ pour un $c > 0$.
	\begin{equation*}
	\begin{split}
	T(n) & = \sqrt{n}T(\frac{n}{\sqrt{n}}) + n\\
	T(n) & \le \sqrt{n}( \frac{cn}{\sqrt{n}}\log (\frac{n}{\sqrt{n}}) ) + n\\
	& = cn\log (\frac{n}{\sqrt{n}}) + n\\
	& = cn\log (n) - cn\log (\sqrt{n}) + n\\
	& = cn\log (n) - cn\log (n^{1/2}) + n\\
	& = cn\log (n) - \frac{1}{2}cn\log n + n\\
	& = cn\log (n) + n\\
	& = O(n\log n)
	\end{split}
	\end{equation*}
	Donc, $O(T(n)) \subseteq O(t(n))$. C'est normal puisqu'on ne peux pas rendre un algorithme diviser pour régner plus rapide en faisant plus de divisions. Plus de divisions, plus d'appels à faire à chaque niveau.
	\section*{Question 2}
	\setcounter{equation}{0})
	\begin{enumerate}
		\item[(a)]
		\begin{verbatim}
turbo(1, 51)
 turbo(1, 48)
  turbo(1, 32)
   turbo(1, 4)
    turbo(1, 2)
    turbo(1, 2)
   turbo(1, 8)
    turbo(1, 2)
    turbo(1, 4)
     turbo(1, 2)
     turbo(1, 2)
  turbo(1, 16)
   turbo(1, 4)
    turbo(1, 2)
     turbo(1, 2)
   turbo(1, 4)
   	turbo(1, 2)
	turbo(1, 2)
 turbo(1, 3)
  turbo(1, 2)
  turbo(1, 1)
		\end{verbatim}
		\item[(b)]
		Récurrence posée:
		\begin{equation*}
		t_k =
		\begin{cases}
		3 \text{ si } k=1\\
		4 + t_{(k-1)} \text{ si } k > 1\\
		\end{cases}
		\end{equation*}
		\item[(c)]
		\setcounter{equation}{0}
		\begin{equation*}
		T(n) =
		\begin{cases}
		1 \text{ si } n=0\\
		3 \text{ si } n=4\\
		4 + T(n/4) \text{ si } n>4 \text{ et n est une puissance de 2 d'une puissance de 2}
		\end{cases}
		\end{equation*}
		

		$$\text{On remplace n par } 2^{2^i}$$
		$$t_i = T(2^{2^i})$$
		$$t_i = 4 + t_{i-1}$$
		$$t_i - t_{i-1} = 4$$
		$$(x-1)(x-1)$$
		$$t_i = C_1\times 1^i + C_2\times i \times 1^i$$
		$$T(2^{2^i}) = t_i \text{ donc,}$$
		$$T(n) = C_1\times 1^{\log_2(\log_2(n))} + C_2\times \log_2(\log_2(n)) \times 1^{\log_2(\log_2(n))}$$
		$$T(n) = C_1 + C_2 \times \log_2(\log_2(n))$$
		$$T(n) \in O(\log(\log(n)))$$
		$$
		\begin{mymatrix}
			1&1&5\\
			1&2&7\\
		\end{mymatrix}
		\rightarrow
		\begin{mymatrix}
		1&1&3\\
		0&1&4\\
		\end{mymatrix}
		\rightarrow
		\begin{mymatrix}
		1&0&-1\\
		0&1&4\\
		\end{mymatrix}
		$$
		$$C_1 = -1, C_2 = 4$$
		$$T(n) = -1 + 4 \log_2(\log_2(n))$$
		$$= 4 \log_2(\log_2(n)) - 1$$
		$$T(n) \in \Theta(\log_2(\log_2(n)) \text{, si n est une puissance de 2 d'une puissance de 2})$$
	\end{enumerate}
	
	\section*{Question 3}
	Turbo engendre 10 multiplications\\
	expoDC engendre 8 multiplications\\
	$a^51$ peut s'effecteur en 6 multiplications, soit:
	\begin{itemize}
		\item $a^2 = a * a$
		\item $a^4 = a^2 * a^2$
		\item $a^{16} = a^4 * a^4$
		\item $a^{17} = a^{16} * a$
		\item $a^{34} = a^{17} * a^{17}$
		\item $a^{51} = a^{34} * a^{17}$
	\end{itemize}

\end{document}